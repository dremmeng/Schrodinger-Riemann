\documentclass[10pt, oneside]{article}
\usepackage{amsmath, amsthm, amssymb, calrsfs, wasysym, verbatim, bbm, color, graphics, geometry, cite}
\geometry{tmargin=.75in, bmargin=.75in, lmargin=.75in, rmargin=.75in}

\newcommand{\R}{\mathbb{R}}
\newcommand{\C}{\mathbb{C}}
\newcommand{\Z}{\mathbb{Z}}
\newcommand{\N}{\mathbb{N}}
\newcommand{\Q}{\mathbb{Q}}
\newcommand{\Cdot}{\boldsymbol{\cdot}}

\newtheorem{thm}{Theorem}
\newtheorem{defn}{Definition}
\newtheorem{conv}{Convention}
\newtheorem{rem}{Remark}
\newtheorem{lem}{Lemma}
\newtheorem{cor}{Corollary}
\newtheorem{example}{Example}
\newtheorem{exe}{Exercise}
\newtheorem{conjecture}{Conjecture}
\newtheorem{remark}{Remark}

\title{From Schrödinger to Riemann: A Quantum-Theoretic Conjecture}
\author{[Drew Remmenga]}

\begin{document}

\maketitle

\begin{abstract}
  We explore a conjectural link between the time-independent Schrödinger equation and the non-trivial zeros of Dirichlet L-functions. By reinterpreting the Basel problem and fractional calculus, we propose a framework where the eigenvalues of a quantum system correspond to the critical zeros of $\zeta(s)$. This work is inspired by Hilbert's vision of unifying quantum mechanics and number theory.
\end{abstract}
\section{Introudction}
The interplay between string theory, number theory, and harmonic analysis has long fascinated physicists and mathematicians, offering a rich tapestry of connections that bridge the abstract and the physical. At the heart of this nexus lies the quest to unify fundamental forces, describe quantum systems, and decode the hidden symmetries of mathematical objects. This paper explores a conjectural link between the time-independent Schrödinger equation and the non-trivial zeros of the Riemann zeta function, drawing inspiration from Hilbert's vision of unifying quantum mechanics and number theory.

String theory, with its reliance on harmonic oscillators and vibrational modes, provides a natural framework for understanding quantum systems. The eigenvalues of these oscillators mirror the discrete energy levels in the Schrödinger equation, while their overtones hint at deeper connections to number-theoretic phenomena. Similarly, number theory---particularly the study of the Riemann zeta function---reveals profound ties to harmonic analysis, as seen in the Basel problem and the distribution of zeta zeros. These zeros, conjectured to lie on the critical line $\Re(s) = \frac{1}{2}$, exhibit a spectral symmetry reminiscent of quantum energy levels.

Harmonic theory further enriches this picture by linking Fourier series, eigenfunctions, and the analytic continuation of zeta. The fractional calculus, employed here to reinterpret the Schrödinger equation, underscores the role of symmetry and normalization in both quantum systems and zeta's analytic structure. By proposing a quantum-theoretic correspondence between the eigenvalues of a confined wave function and the zeros of $\zeta(s)$, this work invites a deeper exploration of how string theory's vibrational paradigms, number theory's enigmatic functions, and harmonic analysis' spectral methods might converge to illuminate the Riemann Hypothesis.

In this spirit, we weave together these disciplines to advance a bold conjecture: that the critical zeros of $\zeta(s)$ emerge as forbidden eigenvalues in a quantum system, their alignment on the critical line a testament to the unity of physics and mathematics.
\section{Schrödinger's Equation and Eigenvalues}
The time-independent Schrödinger equation in one dimension (with $\hbar = 1$) is \cite{Islam1994}:
\begin{align}
  -\frac{1}{2} \frac{d^2 \psi}{dx^2} = E \psi(x), \quad \psi(0) = \psi(\pi) = 0. \label{schrodinger}
\end{align}
The solutions are sinusoidal with eigenvalues $E_n = \frac{n^2}{2}$ for $n \in \N$. The general solution is:
\begin{align}
  \psi(x) = \sum_{n=1}^\infty A_n \sin(nx). \label{eq:wave}
\end{align}

\section{Basel Problem and Zeta Connection}
The Fourier series of $f(x) = x^2$ on $[-\pi, \pi]$ yields:
\begin{align}
  \zeta(2) = \sum_{n=1}^\infty \frac{1}{n^2} = \frac{\pi^2}{6}, \label{eq:basel}
\end{align}
revealing a deep link between $\zeta(s)$ and harmonic analysis. We generalize this to $\zeta(s)$ for $\Re(s) > 1$.

\section{Fractional Calculus and Zeta Zeros}
The Riemann-Liouville fractional integral for $\Re(s) > 0$ is \cite{Hadamard1892} \cite{Hermann2014}:
\begin{align}
  _0D_x^{-s} f(x) = \frac{1}{\Gamma(s)} \int_0^x (x-t)^{s-1} f(t) \, dt. \label{Riemann-Liouville}
\end{align}
Applying this to $|\psi(x)|^2$ and demanding unit normalization suggests:
\begin{align}
  |\sum_{n=1}^\infty \frac{A_n}{n^{\sigma}}|^2 = 1, \quad \sigma = \Re(s). \label{eq:norm}
\end{align}

\section{Conjecture: Quantum Zeta Correspondence}
\begin{conjecture}[Quantum-Theoretic Riemann Hypothesis]
  Let $\psi(x)$ be a solution to \eqref{schrodinger} with eigenvalues $E_n = \frac{n^2}{2}$. If the coefficients $A_n$ are chosen such that:
  \begin{align}
    |\sum_{n=1}^\infty \frac{A_n}{n^s}|^2 = 0, \label{eq:zeta}
  \end{align}
  then the non-trivial solutions $s$ satisfy $\Re(s) = \frac{1}{2}$. This implies an isomorphism between the energy spectrum of $\psi(x)$ and the critical zeros of $\zeta(s)$.
  In this case these values of $s$ are forbidden. There is no way to renormalize the function. Indeed this is true for the trivial zeros and the irremovable pole at $s=1$.
\end{conjecture}

\section{Discussion}
The conjecture posits that:
\begin{itemize}
  \item The normalization condition \eqref{eq:norm} mirrors the analytic continuation of $\zeta(s)$.
  \item The critical line $\Re(s) = \frac{1}{2}$ emerges from the symmetry of the quantum system.
  \item A violation would require a non-unitary or asymmetric $\psi(x)$, akin to a "phase transition" in the zeta zeros.
  \item The half integer spin of the electron (all fermions) proves the Grand Riemann Conjecture. 
\end{itemize}

\bibliographystyle{plain}
\bibliography{references.bib}
\end{document}